%%%%% TIKZ STYLES FOR THESE SLIDES:

\tikzstyle{box}=[rectangle, draw=solarizedBase03, rounded corners, fill=solarizedBlue, drop shadow,
text centered, anchor=north, minimum width = 2cm, minimum height = 1cm, text=white, text width=10cm]

\tikzstyle{plain} = [draw, line width = 2pt, fill=solarizedBase2, text centered, text=solarizedRebase1, draw=solarizedBase2]

%%%%% SLIDE: Give some of the details about the great use of the AVM in problem domains

\begin{frame}
  \frametitle{Missing Features}
  \framesubtitle{\mbox{}}

 \begin{figure}[!htb]
    \centering
    \begin{minipage}{0.5\textwidth}
        \centering
          \input{details/program1_mutation.tex}
        \begin{tikzpicture}
          \path[->]<1> node[plain, text width=20ex]
          (Alive) at (-2.75,-.25) {\Huge \cmarkhide {Alive}};

          \path[->]<2-> node[plain, text width=20ex]
          (Alive) at (-2.75,-.25) {\Huge \cmark {Alive}};
        \end{tikzpicture}
    \end{minipage}%
    \begin{minipage}{0.5\textwidth}
        \centering
        \input{details/program1_mutation.tex}
       \begin{tikzpicture}
          \path[->]<1> node[plain, text width=20ex,
          yshift=0.0in, xshift=1.75in, minimum width=20ex]
          (Killed) {\Huge \xmarkhide {Killed}};

          \path[->]<2> node[plain, text width=20ex,
          yshift=-0.0in, xshift=1.75in, minimum width=20ex]
          (killed) {\Huge \xmarkhide {Killed}};
        \end{tikzpicture}
    \end{minipage}
\end{figure}


\end{frame}
