%%%%% TIKZ STYLES FOR THESE SLIDES:

\tikzstyle{box}=[rectangle, draw=solarizedBase03, rounded corners, fill=solarizedBlue, drop shadow,
text centered, anchor=north, minimum width = 2cm, minimum height = 1cm, text=white, text width=10cm]

\tikzstyle{plain} = [draw, line width = 2pt, fill=solarizedBase2!120, text centered, text=solarizedRebase1, draw=solarizedBase2!120]

%%%%% SLIDE: Give some of the details about the great use of the AVM in problem domains

\begin{frame}
  \frametitle{Missing Features}
  \framesubtitle{\mbox{}}

 \begin{figure}[!htb]
    \centering
    \begin{minipage}{0.5\textwidth}
        \centering
          \begin{tikzpicture}[node distance=0cm, auto,>=stealth, thick]

        \path[line width = 2pt, fill=black!15, draw=solarizedRebase00] (0,0) rectangle (4,-5);
        \draw[solarizedViolet, line width=5pt] (0.25,-0.5) -- (2,-0.5);
        \draw[solarizedOrange, line width=5pt] (0.25,-1) -- (3,-1);
        \draw[solarizedBlue, line width=5pt] (0.25,-1.5) -- (2.5,-1.5);

        \draw[solarizedViolet, line width=5pt] (0.25,-2.5) -- (2,-2.5);
        \draw[solarizedOrange, line width=5pt] (0.25,-3) -- (3.75,-3);
        \draw[solarizedBlue, line width=5pt] (0.25,-3.5) -- (3.5,-3.5);
        \draw[yellow, line width=5pt] (1.25,-3.5) -- (2.75,-3.5);

        \draw[solarizedViolet, line width=5pt] (0.25,-4.5) -- (3,-4.5);


\end{tikzpicture}

        \begin{tikzpicture}
          \path[->]<1> node[plain, text width=20ex]
          (Alive) at (-3.25,-.25) {\Huge \xmarkhide {Data?}};

          \path[->]<2-> node[plain, text width=20ex]
          (Alive) at (-3.25,-.25) {\Huge \xmark {Data?}};
        \end{tikzpicture}
    \end{minipage}%
    \begin{minipage}{0.5\textwidth}
        \centering
        \begin{tikzpicture}[node distance=0cm, auto,>=stealth, thick]

        \path[line width = 2pt, fill=black!15, draw=solarizedRebase00] (0,0) rectangle (4,-5);
        \draw[solarizedViolet, line width=5pt] (0.25,-0.5) -- (2,-0.5);
        \draw[solarizedOrange, line width=5pt] (0.25,-1) -- (3,-1);
        \draw[solarizedBlue, line width=5pt] (0.25,-1.5) -- (2.5,-1.5);

        \draw[solarizedViolet, line width=5pt] (0.25,-2.5) -- (2,-2.5);
        \draw[solarizedOrange, line width=5pt] (0.25,-3) -- (3.75,-3);
        \draw[solarizedBlue, line width=5pt] (0.25,-3.5) -- (3.5,-3.5);
        \draw[yellow, line width=5pt] (1.25,-3.5) -- (2.75,-3.5);

        \draw[solarizedViolet, line width=5pt] (0.25,-4.5) -- (3,-4.5);


\end{tikzpicture}

       \begin{tikzpicture}
          \path[->]<1> node[plain, text width=20ex,
          yshift=0.0in, xshift=1.75in, minimum width=20ex]
          (Killed) {\Huge \xmarkhide {Method?}};

          \path[->]<2> node[plain, text width=20ex,
          yshift=-0.0in, xshift=1.75in, minimum width=20ex]
          (killed) {\Huge \xmark {Method?}};
        \end{tikzpicture}
    \end{minipage}
\end{figure}


\end{frame}
